\documentclass[a4paper]{article}
\usepackage[utf8]{inputenc}
\usepackage[T1]{fontenc}
\usepackage[english]{babel}
\usepackage[hidelinks]{hyperref}
\usepackage{amsmath,amsthm,amssymb}
\usepackage{mathpartir}
\usepackage{macros}

\renewcommand{\labelitemi}{--}

\title{Syntactic regions}
\author{Emmanuel Haucourt \and Samuel Mimram}

\theoremstyle{theorem}
\newtheorem{theorem}{Theorem}
\newtheorem{lemma}[theorem]{Lemma}

\theoremstyle{remark}
\newtheorem{remark}[theorem]{Remark}

\begin{document}
\maketitle

\section{Programs}
We suppose fixed a set $\variables=\set{x,y,\ldots}$ of \emph{variables} and a
set~$\mutexes=\set{a,b,\ldots}$ of \emph{mutexes}. An \emph{expression} $e$ is either
\begin{itemize}
\item a variable~$x$,
\item $\Bfalse$ or $\Btrue$.
\end{itemize}
An \emph{action} is either
\begin{itemize}
\item $\P a$ or $\V a$, for some $a\in\mutexes$, respectively meaning that the
  mutex~$a$ is taken or released,
\item $\pset x\Bfalse$ or $\pset x\Btrue$, for some $x\in\variables$,
\item $\pskip$,
\item $\pdie$ meaning that the program crashes.
\end{itemize}
We write $\actions$ for the set of actions. A \emph{process} is
\[
  P
  \qquad::=\qquad
  A
  \quad|\quad
  \pseq PP
  \quad|\quad
  \por ePP
  \quad|\quad
  \ppar PP
  \quad|\quad
  \ploop eP
\]
with $A\in\actions$.

\section{Operational semantics}
We write~$\B=\set{\Bfalse,\Btrue}$ for the set of \emph{booleans}. An
\emph{environment} is a function $\nu:\variables\to\B$. The \emph{initial
  valuation} $\nu_0$ is such that $\nu_0(x)=\Bfalse$ for every
$x\in\variables$. We abuse notations and write $\nu(\Bfalse)=\Bfalse$ and
$\nu(\Btrue)=\Btrue$, so that $\nu(e)$ is defined for any expression~$e$.

The operational semantics is given by evaluating programs in
environments, \ie judgments of the form $\nu\vdash P$ using the following rules:
\[
  \inferrule{\null}{\penv{\nu}{\pset x\Bfalse}\redto\penv{\nu[x\mapsto\Bfalse]}{\pskip}}
\]
\[
  \inferrule{\penv\nu P\redto\penv{\nu'}{P'}}{\penv{\nu}{\pseq PQ}\redto\penv{\nu'}{\pseq{P'}{Q}}}
  \qquad
  \inferrule{\penv{\nu'}{Q}\redto\penv{\nu''}{Q'}}{\penv{\nu}{\pseq \pskip Q}\redto\penv{\nu''}{Q'}}
\]
\[
  \inferrule{\penv\nu P\redto\penv{\nu'}{P'}\\\nu(e)=\Bfalse}{\penv{\nu}{\por ePQ}\redto\penv{\nu'}{\por{\Bfalse}{P'}Q}}
  \qquad
  \inferrule{\penv\nu P\redto\penv{\nu'}{Q'}\\\nu(e)=\Btrue}{\penv{\nu}{\por ePQ}\redto\penv{\nu'}{\por{\Btrue}P{Q'}}}
\]
\[
  \inferrule{\penv\nu P\redto\penv{\nu'}{P'}}{\penv{\nu}{\ppar PQ}\redto\penv{\nu'}{\ppar{P'}{Q}}}
  \qquad
  \inferrule{\penv\nu Q\redto\penv{\nu'}{Q'}}{\penv{\nu}{\ppar PQ}\redto\penv{\nu'}{\ppar{P}{Q'}}}
\]
\[
  \inferrule{\nu(e)=\Bfalse}{\penv\nu{\pwhile eP}\redto\penv\nu\pskip}
  \qquad
  \inferrule{\penv\nu P\redto\penv{\nu'}{P'}\\\nu(e)=\Btrue}{\penv\nu{\pwhile eP}\redto\penv{\nu'}{\pseq{P'}{\pwhile eP}}}
\]
etc. (we must close under transitivity...)

NB: there is no rule for $\pdie$.

\begin{lemma}[Progress]
  For any crash-free program~$P$ and environment~$\nu$, $\penv\nu P$ reduces to
  $\penv{\nu'}{}$.
\end{lemma}

\begin{lemma}[Determinism]
  For a sequential program there is only one possible derivation tree to the
  normal form~$\penv{\nu'}{}$.
\end{lemma}

\section{Positions}
A \emph{position} is
\[
  \pi
  \qquad::=\qquad
  \pbot
  \quad|\quad
  \ptop
  \quad|\quad
  \pseql \pi
  \quad|\quad
  \pseqr \pi
  \quad|\quad
  \porl\pi
  \quad|\quad
  \porr\pi
  \quad|\quad
  \ppar \pi\pi
  \quad|\quad
  \ploopl\pi
\]
A position~$\pi$ is \emph{valid} for a program~$P$ when the judgment
$\vpos\pi P$ is derivable by the following rules
\begin{align*}
  &
  \inferrule{\null}{\vpos\pbot P}
  &&
  \inferrule{\null}{\vpos\ptop P}
  &&
  \inferrule{\vpos\pi P}{\vpos{\pseql\pi}{\pseq PQ}}
  &&
  \inferrule{\vpos\pi Q}{\vpos{\pseqr\pi}{\pseq PQ}}
  \\
  &
  \inferrule{\vpos\pi P}{\vpos{\porl\pi}{\por xPQ}}
  &&
  \inferrule{\vpos\pi Q}{\vpos{\porr\pi}{\por xPQ}}
  &&
  \inferrule{\vpos\pi P\\\vpos\rho Q}{\vpos{\ppar\pi\rho}{\ppar PQ}}
  &&
  \inferrule{\vpos\pi P}{\vpos{\ploopl\pi}{\ploop xP}}
\end{align*}

% \begin{remark}
  % For loops, we need to have left and right variant of $+$ position because
  % $\pbot+\pbot$ could be a position on the left or on the right (we could do the
  % same for sequence).
% \end{remark}


% where the judgment $\mpos\pi P$ means that $\pi$ is a \emph{maximal position}
% of~$P$, inferred by the following rules
% \begin{align*}
  % ...
% \end{align*}

% \begin{lemma}
  % $\mpos\pi P$ implies $\vpos\pi P$.
  % \end{lemma}

\begin{theorem}
  The set of $P'$ such that $\penv\nu P\redto\penv{\nu'}{P'}$ injects into the
  set of positions (and the injection can be described explicitly).
\end{theorem}

\section{Intervals}
We write $\leq$ for the smallest congruence on positions such that $\pbot\leq\pi$ and $\pi\leq\ptop$ for every position~$\pi$.
\end{document}
